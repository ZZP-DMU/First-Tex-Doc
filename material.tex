\documentclass{ctexart}
\usepackage[margin=1in]{geometry}
\usepackage{amssymb}
\begin{document}
    \section{材料}
    \subsection{概要}
    \paragraph{}水下机器人常用的铝合金是防锈铝合金、硬铝合金、超硬铝合金、锻铝合金,其中防锈铝和锻铝具有较好的塑性、耐磨性和焊接性能,多用于制造耐腐蚀容器和焊接件,适用于制作小型耐压壳体。
    \paragraph{}本AUV选择铝合金7075-T6材料作为耐压壳体材料,其具有较高的力学性能和优良的高温变形能力,能进行各种热加工。为使铝合金之所以有较高的耐腐蚀性,使用阳极氧化于其表面上生产了一层氧化铝薄膜。只要这层氧化膜保持完好,合金便具有良好的抗腐蚀性能。在海洋环境中,AI-Mg合金、AI-Mg-Si合金具有良好的耐腐蚀性能。 \\
    \begin{tabular}{|c|c|c|c|}
    \hline
    材料 & 密度$(kg/m^3)$ & 抗拉强度$(MPa)$ & 屈服强度$(MPa)$ \\ 
    \hline
    铝合金$7075-T6$ & 2810 & 570 & 505 \\
    \hline
    弹性模量$(GPa)$ & 泊松比 & 是否耐腐蚀 & \\
    \hline
    72 & 0.33 & 阳极氧化 & \\
    \hline
    \end{tabular}

    \subsection{耐压壳体材料}
    \paragraph{}本AUV采用鱼雷型整体耐压舱结构,自主水下机器人通常采用模块化设计,将能源导航、控制、载荷、推进等设备分别布置在不同的耐压舱段,通常在舱段之间采用相同的电气、机械接口连接。
    \paragraph{}采用鱼雷型整体耐压舱结构具有以下优点:第一,由于各个舱段之间采用标准接口,整个自主水下机器人更便于实现重构。例如,根据使命的需要可以增加一个附加的能源段来提高续航能力、更换不同的载荷舱段等。第二,由于采用全耐压结构,鱼雷型整体耐压舱结构的自主水下机器人具有容积效率高等特点,且所有设备都布置于干式密封舱内,设备之间的电气连接更为便捷。第三,采用鱼雷型整体耐压舱结构的自主水下机器人可以利用鱼雷制造积累的技术、标准、生产工艺,对于降低生产成本、提高产品质量有一定帮助。
    \begin{enumerate}
        \item 球形壳子强度计算 \\
        球形壳体承受均匀外压时,可以保持其球形而受到均匀压缩,球体厚度满足:
        \[\frac{R_2}{t}>10\]
        式中,$R_2$为耐压壳体中面半径,也就是内径和外径的平均值;$t$为耐压壳体厚度。此时球壳为薄壳,其均匀压力为:
        \[\sigma_1=\sigma_2=\frac{qR_2}{2t}\]
        式中,$\sigma_1$为轴向应力;$\sigma_2$为切向应力;$q$为外压力。
        \item 圆筒壳体强度计算 \\
        当圆筒壳体厚度满足:
        \[\frac{R_2}{t}>10\]
        时,球壳为薄壳,其受外压应力为:
        \[\sigma_1=\frac{qR_2}{2t} \quad \sigma_2=\frac{qR_2}{t}\]
        \item 球形耐压壳体稳定性计算 \\
        根据实际可能最小屈服压力的近似计算公式:
        \[q'=\frac{0.365Et^2}{R_2^2}\]
        式中,$E$为弹性模量。
        \item 圆筒壳体的稳定性分析 \\
        当圆筒为薄壳时,当长度$L$与耐压壳体中面半径,壁厚满足:
        \[L>4.9R_2\sqrt{\frac{R_2}{t}}\]
        时为长圆筒壳体,其屈服压力为:
        \[q'=\frac{1}{4}\frac{E}{(1-u^2)}\frac{t^3}{R_2^3}\]
    \end{enumerate}
    \paragraph{}根据预设的AUV参数,经计算得知该壳体满足500米水深环境的作业需求。

    \subsection{浮力材料}
    \paragraph{}采用新型、轻质、高强度材料SBM-040制作耐压壳体,可以减轻耐压壳体本身的重量,相应地增加其浮力,但只用耐压壳体产生浮力,必然要增大耐压壳体的体积,同时带来耐压壳体重量的增加,也会增大流体阻力。因此,通常在所需容积的耐压壳体产生一定浮力后,其余所需浮力由浮力材料来提供。\\
    \begin{tabular}{|c|c|c|c|c|}
        \hline
        型号&工作水深/m&密度/($g/cm^3$)&吸水率(24h)/\%&压缩强度 \\
        \hline
        SBM-040&500&0.40$\pm$0.02&$\leqslant 1$&$\geqslant 12$ \\
        \hline
    \end{tabular}
    \paragraph{}自主水下机器人上应用最广的是环氧玻璃微珠复合材料,它由热固性树脂和轻质填料混合而成。轻质填料是指玻璃微珠,以5\~{}300 微米的粒径均匀地分散在主体树脂中。成型工艺可采用振动浇注、抽真空浇注、模压等方法。其具有以下特点:
    \begin{enumerate}
        \item 密度低,一般不大于0.7,因为大于此值会使浮力材料大大增加;
        \item 不与水反应,更不溶于水,吸水率低;
        \item 能承受较高的静水压力;
        \item 体积弹性模量与海水相近或略高于海水;
        \item 不可燃且五毒。
    \end{enumerate}
    \paragraph{}以玻璃微球和环氧树脂构成的浮力材料性能优异,易在实验室里制备,容易成形,可以浇注或者机械加成所需的形状。
    \paragraph{}随着深潜器和水下机器人潜深的增加,浮力材料所受外压加大,浮力材料的体积也会有所减小,因而浮力有所降低。其体积收缩率满足以下关系:
    \[e=\frac{\Delta V}{V_0}=-\frac{3p(1-2u)}{E}=-\frac{p}{K}\]
    \[K=\frac{E}{3(1-2u)}\]
    其中$K$为杨氏模量,$\mu$为泊松比。
    \paragraph{}浮力材料的吸水率也与外压有关。当使用压力大约为浮力材料强度的60
                \%时,吸水率急剧增加。另外,吸水率的增加也大致与表面积、加压时间
                成正比。这些因素会使浮力材料所产生的浮力有所改变。浮力材料吸水率满足以下关系:
    \[\eta=\frac{\Delta M}{M_0}=\frac{M_1-M_0}{M_0} \times 100\%\]
    式中$M_0$为浮力块的初始质量,$M_1$为浮力块的最终质量。
\end{document}