\documentclass{ctexart}
\usepackage[margin=1in]{geometry}
\begin{document}
    \section{环境、设备安全和法律法规}
    \subsection{能源动力部分}
    \paragraph{设备安全方面} 
        本AUV电池舱模块设计为可自动拆卸形式,目的是在紧急情况下可以将电池模块卸载掉,从而减小重力,使其上浮,保证设备其余部分的安全。此外,本AUV还搭载了应急电源。
        若主系统发生故障或失去电源,有应急能源提供备用电源,为应急安全装置供电。
    \paragraph{海洋环境方面}
        本AUV的电池模块使用的是锂电池,锂电池中不含镉、铅、汞等对环境有害的元素,具有无公害,无记忆效应等特点,对海洋环境可能造成的污染在可控范围内。
    \subsection{密封安全部分}
    \paragraph{设备安全方面} 
        对于本AUV的耐压壳体我们选取了合适的O型密封圈实现耐压舱封头与壳体的连接及密封,使耐压壳体的泄漏量在工作压力下可以近似为零。保证了耐压舱内装有主控制电路板及各类传感器等价值不菲的核心元件。对于水下推进器,我们选取了旋转机械动密封,该结构可靠,在各种动密封中泄漏量可以限制到很少,只要主密封面的表面粗糙度和平直度能保证达到要求,可以达到很少泄漏量,甚至肉眼看不见泄漏。确保了海水不会渗透进推进系统中污染液压油等工作介质,让水下机器人失去动力,在水下失踪。
    \paragraph{海洋环境方面}  
        本AUV的推进系统采用旋转机械动密封,适用于海底的高压力,易腐蚀的环境,保证了水下机器人内部容易污染海洋环境的介质不会外泄。
    \subsection{材料部分}
    \paragraph{}浮力材料对于人员安全与海洋环境方面的考虑:
    \begin{enumerate}
        \item 能承受高的静水压力,根据目前国产浮力材料生产工艺,H25HS体积分数在68\%以下时,适合的成型方法是浇注成型,可以获得安全使用深度大于6 000 m,密度不大于0.58 $g/cm^3$的固体浮力材料;H25HS体积分数为68\%~70\%时,真空捣打成型和等静压成型方法最优,能够获得安全使用深度为2 000~4 000 m,密度0.48~0.52 $g/cm^3$的固体浮力材料,吸水率不超过2\%,可以耐受40Mpa的压力,能够完全胜任水深500m作业需求;
        \item 不会与水反应,更不溶于水,不会污染海洋环境;
        \item 该浮力材料不可燃,且无毒,在陆地环境下消防隐患小。
    \end{enumerate}
    \paragraph{}耐压壳体的设计考虑:
    \begin{enumerate}
        \item 经过校核,本AUV所采用的铝合金7075-T6耐压壳体能够满足500m水下作业需求;
        \item 所有设备都布置于干式密封舱内,不仅设备之间的电气连接更为便捷,还降低了内外泄露的风险。
    \end{enumerate}
\end{document}